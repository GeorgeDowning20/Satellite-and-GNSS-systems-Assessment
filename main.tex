\documentclass[11pt]{article}
\usepackage[colorlinks=true, allcolors=blue]{hyperref}
\usepackage[english]{babel}
\usepackage[numbers]{natbib}
\usepackage{url}
\usepackage[utf8x]{inputenc}
\usepackage{amsmath}
\usepackage{graphicx}
\usepackage{parskip}
\usepackage{fancyhdr}
\usepackage{vmargin}
\usepackage{tikz}
\usepackage{pgfplots} 
\usepackage{minted}
\usepackage{placeins}
\usepackage{tikz}
\usepackage[compatibility,siunitx,  americanvoltages, americancurrents, europeanresistors, europeaninductors, americanports,%
straightlabels, fetbodydiode, straightvoltages]{circuitikz}
\usepackage{amssymb}





\usepackage[compatibility,siunitx,  americanvoltages, americancurrents, europeanresistors, europeaninductors, americanports,%
  straightlabels, fetbodydiode, straightvoltages]{circuitikz}

\usepackage{tikz,amsmath, amssymb,bm,color,pgfkeys,siunitx,ifthen,ulem}
\usepackage{pgfplots}

%\pgfplotsset{compat=1.14}
\usetikzlibrary{shapes,arrows}
%\usepackage{agaramondc}					% Adobe Garamond, custom shape
%\renewcommand{\shapedefault}{rtl} % rtl: roman tabular lining

\makeatletter

%% bandstop filter (adapted from highpass)
\pgfcircdeclarebipole{}{\ctikzvalof{bipoles/highpass/width}}{*bandstop}{\ctikzvalof{bipoles/highpass/width}}{\ctikzvalof{bipoles/highpass/width}}{
	\pgf@circ@res@step = \ctikzvalof{bipoles/highpass/width}\pgf@circ@Rlen
	\divide \pgf@circ@res@step by 2
	
	\pgfpathmoveto{\pgfpoint{\pgf@circ@res@left}{\pgf@circ@res@zero}}
	\pgf@circ@res@other = \pgf@circ@res@left
	\advance\pgf@circ@res@other by \pgf@circ@res@step 
	
	\ifpgf@circuit@dashed
	\pgfsetdash{{0.1cm}{0.1cm}}{0cm} 
	\fi	
	
	% draw outer box
	\pgfsetlinewidth{\pgfkeysvalueof{/tikz/circuitikz/bipoles/thickness}\pgfstartlinewidth}
	\pgfpathrectanglecorners{\pgfpoint{\pgf@circ@res@left}{\pgf@circ@res@up}}{\pgfpoint{\pgf@circ@res@right}{\pgf@circ@res@down}}
	\pgfusepath{draw}
	
	\ifpgf@circuit@inputarrow
	{
		\advance \pgf@circ@res@left by -.5\pgfkeysvalueof{/tikz/circuitikz/bipoles/thickness}\pgfstartlinewidth
		\pgftransformshift{\pgfpoint{\pgf@circ@res@left}{0pt}}
		\pgfnode{inputarrow}{tip}{}{pgf@inputarrow}{\pgfusepath{fill}}
	}
	\fi
	
	% rotate inner symbol
	\def\pgfcircmathresult{\expandafter\pgf@circ@stripdecimals\pgf@circ@direction\pgf@nil}
	\ifnum \pgfcircmathresult > 45 \ifnum \pgfcircmathresult < 135
	\pgftransformrotate{270}
	\fi\fi
	\ifnum \pgfcircmathresult > 134 \ifnum \pgfcircmathresult < 225  % 134 degree, because >= 135 is not possible
	\pgftransformrotate{180}
	\fi\fi
	\ifnum \pgfcircmathresult > 224 \ifnum \pgfcircmathresult < 315
	\pgftransformrotate{90}
	\fi\fi
	
	% draw inner symbol
	\pgfsetdash{}{0pt}	% always draw solid line for inner symbol
	\pgfsetarrows{-} %never draw arrows
	\pgfsetlinewidth{\pgfstartlinewidth}
	\pgfpathmoveto{\pgfpoint{-0.5\pgf@circ@res@step}{0.5\pgf@circ@res@step}}
	\pgfpathsine{\pgfpoint{.25\pgf@circ@res@step}{.25\pgf@circ@res@step}}
	\pgfpathcosine{\pgfpoint{.25\pgf@circ@res@step}{-.25\pgf@circ@res@step}}
	\pgfpathsine{\pgfpoint{.25\pgf@circ@res@step}{-.25\pgf@circ@res@step}}
	\pgfpathcosine{\pgfpoint{.25\pgf@circ@res@step}{.25\pgf@circ@res@step}}
	\pgfusepath{draw}
	
	\pgfpathmoveto{\pgfpoint{-0.5\pgf@circ@res@step}{0}}
	\pgfpathsine{\pgfpoint{.25\pgf@circ@res@step}{.25\pgf@circ@res@step}}
	\pgfpathcosine{\pgfpoint{.25\pgf@circ@res@step}{-.25\pgf@circ@res@step}}
	\pgfpathsine{\pgfpoint{.25\pgf@circ@res@step}{-.25\pgf@circ@res@step}}
	\pgfpathcosine{\pgfpoint{.25\pgf@circ@res@step}{.25\pgf@circ@res@step}}
	\pgfusepath{draw}
	\pgfpathmoveto{\pgfpoint{-0.15\pgf@circ@res@step}{-0.15\pgf@circ@res@step}}
	\pgfpathlineto{\pgfpoint{0.15\pgf@circ@res@step}{0.15\pgf@circ@res@step}}
	\pgfusepath{draw}
	
	\pgfpathmoveto{\pgfpoint{-0.5\pgf@circ@res@step}{-0.5\pgf@circ@res@step}}
	\pgfpathsine{\pgfpoint{.25\pgf@circ@res@step}{.25\pgf@circ@res@step}}
	\pgfpathcosine{\pgfpoint{.25\pgf@circ@res@step}{-.25\pgf@circ@res@step}}
	\pgfpathsine{\pgfpoint{.25\pgf@circ@res@step}{-.25\pgf@circ@res@step}}
	\pgfpathcosine{\pgfpoint{.25\pgf@circ@res@step}{.25\pgf@circ@res@step}}
	\pgfusepath{draw}
	%	\pgfpathmoveto{\pgfpoint{-0.15\pgf@circ@res@step}{-0.65\pgf@circ@res@step}}
	%	\pgfpathlineto{\pgfpoint{0.15\pgf@circ@res@step}{-0.35\pgf@circ@res@step}}
	%	\pgfusepath{draw}
}

\tikzset{
	*bandstop/.style={\circuitikzbasekey, /tikz/to path=\pgf@circ@*bandstop@path},
}
\def\pgf@circ@*bandstop@path#1{\pgf@circ@bipole@path{*bandstop}{#1}}




\makeatother


\usetikzlibrary{backgrounds,calc,positioning}

\usetikzlibrary{circuits.ee.IEC}
\usetikzlibrary{arrows}


% sym32a style

\ctikzset{tripoles/mos style/arrows}
\ctikzset{
	/tikz/circuitikz/quadpoles/coupler/width=1,%1.3
	/tikz/circuitikz/quadpoles/coupler/height=0.952,%1.3
	/tikz/circuitikz/quadpoles/coupler2/width=1,%1.3
	/tikz/circuitikz/quadpoles/coupler2/height=0.952,%1.3
	/tikz/circuitikz/quadpoles/transformer/width=1.425,%1.5
	/tikz/circuitikz/quadpoles/transformer/height=1.425,%1.5
	/tikz/circuitikz/quadpoles/transformer core/width=1.425,%1.5
	/tikz/circuitikz/quadpoles/transformer core/height=1.425,%1.5
	/tikz/circuitikz/quadpoles/gyrator/width=1.425,%1.5
	/tikz/circuitikz/quadpoles/gyrator/height=1.425,%1.5
	%/tikz/circuitikz/monopoles/tlinestub/width=0.1875,%0.25 no effect!
	/tikz/circuitikz/tripoles/american and port/height=0.95,%.8
	/tikz/circuitikz/tripoles/american nand port/height=0.95,%.8
	/tikz/circuitikz/tripoles/american or port/height=0.95,%.8
	/tikz/circuitikz/tripoles/american nor port/height=0.95,%.8
	/tikz/circuitikz/tripoles/american xor port/height=0.95,%.8
	/tikz/circuitikz/tripoles/american xnor port/height=0.95,%.8
	/tikz/circuitikz/bipoles/tline/height=0.4,%0.3
%	/tikz/circuitikz/bipoles/tline/width=1.2,%0.8
	/tikz/circuitikz/bipoles/diode/height=0.375,%
	/tikz/circuitikz/bipoles/diode/width=0.375,%
	/tikz/circuitikz/bipoles/varcap/height=0.375,%
	/tikz/circuitikz/bipoles/varcap/width=0.375,%
	/tikz/circuitikz/tripoles/triac/height=1.05,%
	/tikz/circuitikz/tripoles/triac/width=0.952,%
	/tikz/circuitikz/tripoles/thyristor/height=1.05,%
	/tikz/circuitikz/tripoles/thyristor/width=0.952,%
	/tikz/circuitikz/tripoles/op amp/height=0.952,%
	/tikz/circuitikz/tripoles/op amp/width=1.2,%
	/tikz/circuitikz/tripoles/op amp/font=\footnotesize,
	/tikz/circuitikz/tripoles/gm amp/height=0.952,% 1.7
	/tikz/circuitikz/tripoles/gm amp/width=1.2,% 1.4
	%	/tikz/circuitikz/tripoles/gm amp/font=\footnotesize,
	/tikz/circuitikz/tripoles/plain amp/height=0.952,% 1.7
	/tikz/circuitikz/tripoles/plain amp/width=1.2,% 1.4
	/tikz/circuitikz/bipoles/resistor/voltage/straight label distance/.initial=.8,
	/tikz/circuitikz/bipoles/generic/voltage/straight label distance/.initial=.8,
	/tikz/circuitikz/bipoles/inductor/voltage/straight label distance/.initial=.8,
	/tikz/circuitikz/bipoles/fullgeneric/voltage/straight label distance/.initial=.8,
	/tikz/circuitikz/bipoles/capacitor/voltage/straight label distance/.initial=1.0,
	/tikz/circuitikz/bipoles/thickness=1.6,
}
\ctikzset{v/.append style={/tikz/european voltages}}

\definecolor{netlabelcolor}{rgb}{0, 0, 0.25}
\definecolor{lttotitextcolor}{rgb}{0, 0.4, 0.25}
\definecolor{lttotidrawcolor}{rgb}{0.6, 0.6, 0.6}
\definecolor{netcolor}{rgb}{0, 0, 0.5}

\pgfkeys{/lt2ti/netlabel/font/.initial= \small}
\pgfkeys{/lt2ti/text/font/.initial= \small}

\pgfkeys{/lt2ti/Net/.style= {netcolor}}
\tikzstyle{dashdotdotted}=[dash pattern=on 3pt off 2pt on \the\pgflinewidth off 2pt on \the\pgflinewidth off 2pt]

\pgfkeys{/lt2ti/VArrow/.style= {->,>=latex}}
\pgfkeys{/lt2ti/SArrow/.style= {->,>=angle 90}}
%%%%%%%%%%%%%%%%%%%%%%%%%%%%%%%%%%%%%%%%%%%%%%%%%%%%%%%%%%%%%%%%%%%%%%%%%%%%%%%


\def\WSPath{/home/g/Uni code/Electronic Processing and Communications (EEEE2044 UNUK)/EEEE2044 Coursework 1 - Digital Electronics/}
\def\LTSpicePath{/home/g/.var/app/com.usebottles.bottles/data/bottles/bottles/UNI/drive_c/Program Files/LTC/LTspiceXVII/lib/sym}
\def\LT2Tikzscript{tools/lt2circuitikz-master/lt2ti.py}

\newcommand{\LTSpice}[1]{\input{|python3 \LT2Tikzscript '\WSPath#1'}}

%%%%%%%%%%%%%%%%%%%%%%%%%%%%%%%%%%%%%%%%%%%%%%%%%%%%%%%%%%%%%%%%%%%%%%%%%%%%%%%
\title{Satellite and GNSS systems}								
\author{George Downing}								                        
\date{March 9, 2022}					

\graphicspath{{images/}}
\setmarginsrb{3 cm}{2.5 cm}{3 cm}{2.5 cm}{1 cm}{1.5 cm}{1 cm}{1.5 cm}

\usetikzlibrary{angles,quotes} 
\usetikzlibrary{arrows.meta}
\usetikzlibrary{calc}
\ctikzset{tripoles/mos style/arrows} 		
\usetikzlibrary{matrix,calc}				            

\makeatletter
\let\thetitle\@title
\let\theauthor\@author
\let\thedate\@date
\makeatother

\pagestyle{fancy}
\fancyhf{}
\rhead{\theauthor}
\lhead{\thetitle}
\cfoot{\thepage}

\newcommand{\Q}{Q1} 

\newenvironment{conditions}[1][where:]
  {#1 \begin{tabular}[t]{>{$}l<{$} @{${}={}$} l}}
  {\end{tabular}\\[\belowdisplayskip]}
\begin{document}

%%%%%%%%%%%%%%%%%%%%%%%%%%%%%%%%%%%%%%%%%%%%%%%%%%%%%%%%%%%%%%%%%%%%%%%%%%%%%%%%%%%%%%%%%
                                    % Title Page
%%%%%%%%%%%%%%%%%%%%%%%%%%%%%%%%%%%%%%%%%%%%%%%%%%%%%%%%%%%%%%%%%%%%%%%%%%%%%%%%%%%%%%%%%

\begin{titlepage}
    \centering
    %\vspace*{0.5 cm}
    \includegraphics[scale = 0.4]{Config/uon.png}\\[1.0 cm]	% University Logo

    
    \textsc{\Large Department of Electrical and Electronic Engineering
    Faculty of Engineering}\\[1.5 cm]	% University Name
    \textsc{\large Contemporary Engineering Themes B }\\[0.5 cm]				% Course Code
    \textsc{\large (EEEE2064 UNUK) (SPR1 22-23)}\\[0.5 cm]				% Course Name

    \rule{\linewidth}{0.2 mm} \\[0.4 cm]
    { \huge \bfseries \thetitle}\\
    \rule{\linewidth}{0.2 mm} \\[1.5 cm]

    \begin{minipage}{0.4\textwidth}
        \begin{flushleft} \large
            \emph{Author:}\\
            \theauthor
        \end{flushleft}
    \end{minipage}~
    \begin{minipage}{0.4\textwidth}
        \begin{flushright} \large
            \emph{Student Number:} \\
            20273662									% Your Student Number
        \end{flushright}
    \end{minipage}\\[1.5 cm]

    {\large \thedate}\\[0 cm]

    \vfill

\end{titlepage}
\pagebreak

%%%%%%%%%%%%%%%%%%%%%%%%%%%%%%%%%%%%%%%%%%%%%%%%%%%%%%%%%%%%%%%%%%%%%%%%%%%%%%%%%%%%%%%%%
                                    % Table of Contents
%%%%%%%%%%%%%%%%%%%%%%%%%%%%%%%%%%%%%%%%%%%%%%%%%%%%%%%%%%%%%%%%%%%%%%%%%%%%%%%%%%%%%%%%%

\tableofcontents
\pagebreak

%%%%%%%%%%%%%%%%%%%%%%%%%%%%%%%%%%%%%%%%%%%%%%%%%%%%%%%%%%%%%%%%%%%%%%%%%%%%%%%%%%%%%%%%%
                                    % Part 1
%%%%%%%%%%%%%%%%%%%%%%%%%%%%%%%%%%%%%%%%%%%%%%%%%%%%%%%%%%%%%%%%%%%%%%%%%%%%%%%%%%%%%%%%%

\section{Part 1}\label{part1}

Design a satellite communication link operating in the Ku band to meet C/N and link margin specifications.


\subsection{Satellite Parameters}\label{Satellite Parameters}

\begin{itemize}
    \item Antenna gain 25 dB
    \item Receive system noise temperature 500 K
    \item Transponder saturated output power in Ku band 40 W
    \item Transponder bandwidth 36 MHz
    \item Signals: FM-TV analog signal
    \item Earth station receiver IF noise bandwidth is 27MHz
    \item Minimum C/N overall = 12 dB
    \item Boltzmann’s constant in decibels is k=-228.6dBW/K/Hz
\end{itemize} 

\subsection{UPLINK}\label{UPLINK}

\subsubsection{Q:}
UPLINK: Design a transmitting earth station (transmitted antenna gain in dB and earth station transmitted power in W) to provide (C/N) up of 35 dB in a Ku-band transponder. Use an uplink antenna with a diameter of 3m and an aperture efficiency of 65\%. The uplink station is located at -2 dB contour of the satellite footprint. Allow 1.5 dB for clear air atmospheric attenuation and other losses. Path length to satellite is 38 500 km. Assume standard frequency allocation of 14GHz for the uplink in Ku-band.


\subsubsection{A:}

\begin{scriptsize}
    \inputminted{matlab}{Project/Matlab/UPLINK.m}
\end{scriptsize}

\begin{equation}
    \begin{split}
        \text{tx}>=&34.2787W\\
        \text{antenna} =& 78.6769dB\\
    \end{split}
\end{equation}
\subsection{DOWNLINK}\label{DOWNLINK}

\subsubsection{Q:}

DOWNLINK: Find the power level of the earth station receiver and the antenna gain at the earth receiver station so that overall carrier to noise ratio is 15 dB. Miscellaneous downlink losses are 0.5dB. Earth station is located at -2dB contour of satellite transmitting antenna. The earth station receiver has the following noise temperatures: noise temperature of the input signal is 25K, noise temperature of the RF amplifier is 400K, noise temperature of the mixer is 450 K and the noise temperature of the IFamplifier is 550K. The gain of the RF amplifier is 35 dB, the gain of the mixer is 0dB and the gain of the IF amplifier is 20dB. Assume standard frequency allocation of 11GHz for the downlink in Ku-band.

\subsubsection{A:}

\begin{scriptsize}
    \inputminted{matlab}{Project/Matlab/DOWNLINK.m}
\end{scriptsize}

% Ku down = 11Ghz
\begin{equation}
    \begin{split}
        \text{Received Power} =& -74.2855dBW\\
        \text{Antenna Gain} =& 92.1727dB\\
    \end{split}
\end{equation}
%%%%%%%%%%%%%%%%%%%%%%%%%%%%%%%%%%%%%%%%%%%%%%%%%%%%%%%%%%%%%%%%%%%%%%%%%%%%%%%%%%%%%%%%%
                                    % Part 2
%%%%%%%%%%%%%%%%%%%%%%%%%%%%%%%%%%%%%%%%%%%%%%%%%%%%%%%%%%%%%%%%%%%%%%%%%%%%%%%%%%%%%%%%%


\section{Part 2}\label{part2}

\begin{equation}
    \begin{split}
        ID &= 20273662\\
        \therefore K &= 2 + 2 = 4\\
    \end{split}
\end{equation}

\subsection{1}
\subsubsection{Qa}
A GPS signal is transmitted from an altitude of 20,000 km above the earth’s surface. Calculate the path length assuming an elevation angle of (50 + K) degrees.
\subsubsection{Aa}

\begin{equation}
\begin{split}
    R\ &=\ \sqrt{R_{E}^{2}\left(\sin^{2}\left(\theta_{EL}\right)-1\right)+R_{SV}^{2}-R_{E}\sin\left(\theta_{EL}\right)}\\
    &=\ \sqrt{\left(6378.1^{2}\cdot\left(\sin^{2}\left(52\right)-1\right)\right)+20000^{2}-\left(6378.1\cdot\sin\left(52\right)\right)}\\
    &= 19288 km
\end{split}
\end{equation}


\subsubsection{Qb}
How many GNSS satellites are required to achieve a navigation fix in 3 dimensions and why?
\subsubsection{Ab}

4 satellites. 3 satellites would resolve to 2 possible points in space. If the device was fitted with an atomic clock, this would allow it to work out out which one is correct. However most devices do not have an atomic clock, so a fourth satellite is required.

\subsubsection{Qc}
What is the main difference between how CDMA is used in communication systems and GNSS systems?
\subsubsection{Ac}

communication-based systems are typically synchronous CDMA using orthogonal codes, whereas GNSS uses asynchronous CDMA using pseudo-random noise codes.
\subsubsection{Qd}
If a GPS C/A code receiver is turned on with no knowledge of its location or any other information provided to it, how long does it typically take to get a navigation fix and why?
\subsubsection{Ad}

around 15 minutes because it needs to download the ephemerids and almanacs, which contain information about the satellite's orbit and position that is required to perform trilateration. Because the download link is slow, this can take a while, however, is needed before the satellites can be used.

\subsection{2}

\subsubsection{Qa}
In no more than 200 words describe the fundamental concept of how a GNSS system works and any technology which enables it?
\subsubsection{Aa}

Atomic clocks are kept on satellites that keep time very accuratly. Each satellite transmits ephemerids and almanacs that encode the trajectory information of that satellite. The satellites use asynchronous CMDA using pseudo-random noise codes, allowing multiple satellites to transmit simultaneously and identify which signal is from which satellite. 

Each satellite transmits a pulse that is received by the receiver. The distance between the satellite and the receiver can be calculated using  \eqref{eq1}.

\begin{equation}\label{eq1}
    \text{distance} = \text{speed of light} \cdot \text{time}
\end{equation}


Trilateration is used to estimate the position of a receiver given distance estimations from at least three known points, as shown in \eqref{eqref2}.

\begin{equation}
    \begin{split}
        d1 &= c \Delta t_{12} = \sqrt{\left(x_1 - x\right)^2 + \left(y_1 - y\right)^2 + \left(z_1 - z\right)^2}\\
        d2 &= c \Delta t_{23} = \sqrt{\left(x_2 - x\right)^2 + \left(y_2 - y\right)^2 + \left(z_2 - z\right)^2}\\
        d3 &= c \Delta t_{31} = \sqrt{\left(x_3 - x\right)^2 + \left(y_3 - y\right)^2 + \left(z_3 - z\right)^2}\\
    \end{split}\label{eqref2}
\end{equation}

Where multiple satellites are used with different covariances, a Kalman filter can combine all the data to find the best possible estimate of position. 


\subsubsection{Qb}
What is the length (period) of a chip for a the GPS C/A code signal? How much is that in metres?
\subsubsection{Ab}

 1023 chips per ms $\therefore$ 0.977 \mu s per chip.

\begin{equation}
  \begin{split}
      \text{distance} &= \text{speed of light} \cdot \text{time}\\
        &= 299 792 458 \cdot 0.977e-6\\
        &= 292.9 m
  \end{split}
\end{equation}


\subsection{3}

\subsubsection{Qa}
What is the main lobe(s) bandwidth for a GNSS signal modulated with BPSK(K)? and BOC(1,1)?
\subsubsection{Aa}

\begin{equation}
    \begin{split}
        f_s & = m \cdot 1.023 \cdot 10^6\\
        f_c & = n \cdot 1.023 \cdot 10^6\\
    \end{split}
\end{equation}

\begin{equation}
    \begin{split}
        G^{0}_{BPSK}\left(f_c\right) &= f_{c}\left[\frac{\sin\left(\frac{\pi x}{f_{c}}\right)}{\left(\frac{\pi x}{f_{c}}\right)}\right]^{2}\\
        G^{0}_{BOC}\left(f_s,f_c\right) &= f_{c}\left[\frac{\sin\left(\frac{\pi x}{f_{c}}\right)\cdot\sin\left(\frac{\pi x}{2f_{s}}\right)}{\pi x\cos\left(\frac{\pi x}{2f_{s}}\right)}\right]^{2}\\
    \end{split}
\end{equation}


For BPSK(2), the main lobe bandwidth is the carrier frequency for L1 band: 1575.42 MHz. The bandwidth is 8 * chipping rate = 8.184Mhz

for BOC(1,1), the two main lobes are located at 1574.661Mhz and 1576.179Mhz. the bandwidth of the two main lobes is 2 * chipping rate = 2.046Mhz
\subsubsection{Qb}
GPS satellite is moving with a line of sight relative velocity of (K×400)m/s to a receiver.  What is the Doppler shift to the centre frequency caused by this movement on the L1 and L2 signals from the satellite at the receiver?
\subsubsection{Ab}


\begin{equation}
    \begin{split}
        L1 &= 1575.42 \cdot 10^6\\
        L2 &= 1227.60 \cdot 10^6\\
        c &= 299 792 458\\
        f'&=\frac{f_{0}}{\left(1+\frac{v}{c}\right)}\\
    \end{split}
\end{equation}

\begin{equation}
    \begin{split}
        v &= 4*-400\text{m/s}\\
        f'_{L1} &=\frac{1575.42 \cdot 10^6}{\left(1+\frac{4\cdot-400}{299 792 458}\right)}\\
        f'_{L1} &= 1575.428392 \cdot 10^6\ \text{Hz}\\
        \therefore\ f'_{L1}-f_0 &= + 8392 \text{Hz}\\
    \end{split}
\end{equation}

Hence an increase in the frequency of 8392Hz as the satellite moves towards the receiver.

\begin{equation}
    \begin{split}
        v &= 4*-400\text{m/s}\\
        f'_{L2} &=\frac{1227.60 \cdot 10^6}{\left(1+\frac{4\cdot-400}{299 792 458}\right)}\\
        f'_{L2} &= 1227.606552 \cdot 10^6\ \text{Hz}\\
        \therefore\ f'_{L2}-f_0 &= + 6552 \text{Hz}\\
    \end{split}
\end{equation}

Hence an increase in the frequency of 6552Hz as the satellite moves towards the receiver.

\subsubsection{Qc}
A 10 MHz TXCO is driving a receiver’s front end and has a frequency deviation of 4 ppm.  It drives a direct downconversion RF front-end mixing the L2 frequency.  What is the frequency offset at baseband due to the TXCO?
\subsubsection{Ac}

\begin{equation}
    \begin{split}
        L2 &= 1227.60 \cdot 10^6\\
        ppm &= 4\\
        f_{\text{offset}} &= 1227.60 \cdot 10^6 \cdot 4^{-6} \\
        &= 4.910 \text{kHz}
    \end{split}
\end{equation}

\subsection{4}

\subsubsection{Q}
What are the significant error sources of a GNSS and how they might be mitigated?  Use 200 words or less, a bullet point list with descriptions is acceptable/preferred
\subsubsection{A}

\begin{itemize}
    \item satellites position knowledge: use ephemeris to provide precise orbital trajectory information.
    \item Ionospheric and Tropospheric delays: the majority is Ionosphere: corrected by transmitting on two frequency's and using the difference in the delay to make corrections.
    \item Differential GPS (mitigation): DGPs and RTK: DGPs to correct for Ionospheric and Tropospheric delays. RTK additionally uses kinematics to resolve to 1cm accuracy.
    \item Tracking Errors: Taking long averages and using strong signals can help mitigate this. platform movement can cause tracking errors, where averaging is not appropriate, the sensors can be fused to an inertial system for better accuracy.
    \item multipathing: This can be mitigated using high bandwidth. Additional receiver mounting considerations can be used to reduce the effect of multipathing.
    \item blockage and attenuation:  having an unobstructed view of the sky helps mitigate this. If the receiver is next to tall buildings, the GDOP is reduced due to bad geometry. consider mounting on a pole. the moisture content of the air (clouds) can also cause attenuation. 
    \item number of satellites: more satellites means better accuracy. - use multiband receivers. UBlox F9P really cool. 40+ satellites
    \item high receiver noise: poor antenna(low gain or out of tune) or poor receiver: increase $C/N_0$ hence fewer sats. use better equipment.
\end{itemize}

%%%%%%%%%%%%%%%%%%%%%%%%%%%%%%%%%%%%%%%%%%%%%%%%%%%%%%%%%%%%%%%%%%%%%%%%%%%%%%%%%%%%%%%%%
                                    % References
%%%%%%%%%%%%%%%%%%%%%%%%%%%%%%%%%%%%%%%%%%%%%%%%%%%%%%%%%%%%%%%%%%%%%%%%%%%%%%%%%%%%%%%%%

\addcontentsline{toc}{section}{References}
\bibliographystyle{IEEEtranN}
\bibliography{biblist}

%%%%%%%%%%%%%%%%%%%%%%%%%%%%%%%%%%%%%%%%%%%%%%%%%%%%%%%%%%%%%%%%%%%%%%%%%%%%%%%%%%%%%%%%%
                                    % Appendix
%%%%%%%%%%%%%%%%%%%%%%%%%%%%%%%%%%%%%%%%%%%%%%%%%%%%%%%%%%%%%%%%%%%%%%%%%%%%%%%%%%%%%%%%%
\pagebreak
% \appendix
% \section*{Appendices}\addcontentsline{toc}{section}{Appendices}\label{appendix:main}
% \renewcommand{\thesubsection}{\Alph{subsection}}

% \subsection{Matlab Code}

% \subsection{LTSpice Code}

%%%%%%%%%%%%%%%%%%%%%%%%%%%%%%%%%%%%%%%%%%%%%%%%%%%%%%%%%%%%%%%%%%%%%%%%%%%%%%%%%%%%%%%%%
\end{document}




